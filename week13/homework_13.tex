\documentclass{article}
\usepackage{amsfonts}
\usepackage{amsmath}
\usepackage{hyperref}
\usepackage[left=2cm,right=2cm,top=2cm,bottom=2cm]{geometry}
%\newenvironment{amatrix}[1]{%
%  \left(\begin{array}{@{}*{#1}{c}|c@{}}
%}{%
%  \end{array}\right)
%}

\author{Mark Vijfvinkel \& Aram Verstegen \\ 0863002(s134674), 100863(s4092368) \\ Radboud University}
\title{Cryptography 1, Homework 13}

\begin{document}
\maketitle
\date

\section{Question 1: Fermat Test}

We start with $a = 2$.
$gcd(2, 263) = 1$ \\
$a^{n-1} \not\equiv 1 \pmod n = 2^{262} \equiv 1 \pmod{263}$\\
Thus for $a = 2$ there is no composition possible.\\
\noindent
We continue with $a = 3$.
$gcd(3, 263) = 1$ \\
$a^{n-1} \not\equiv 1 \pmod n = 3^{262} \equiv 1 \pmod{263}$\\
Thus for $a = 3$ there is no composition possible.\\
\noindent
We continue with $a = 5$.
$gcd(5, 263) = 1$ \\
$a^{n-1} \not\equiv 1 \pmod n = 5^{262} \equiv 1 \pmod{263}$\\
Thus for $a = 5$ there is no composition possible.\\
\noindent
This means that for $k = 3$, $n = 263$ is probably prime or Carmichael.

\section{Question 2: Miller-Rabin Test}
%TODO needs checking!
If we follow the algorithm given in the lecture, we first have to factor $n - 1 = 263 - 1 = 262$.
Thus, $262 = 2 * 131$, since $131$ is prime we cannot factor this any further.
Our $r = 1$ and $t = 131$
\noindent
First we have to test for $gcd(a,n) \not = 1$, $gcd(2,263) = 1$.
Then we test for $a^t \not\equiv \pm 1 \pmod{n}$, $2^{131} \equiv 1 \pmod{263}$. So we can skip the next loop.\\
\noindent
Next we test $gcd(3,263) = 1$ and $3^{131} \equiv 1 \pmod{263}$. Again we can skip the next loop.\\
\noindent
Next we test $gcd(5,263) = 1$ and $5^{131} \equiv 262 \pmod{263}$. Now we should enter the next loop, but notice that $r - 1 = 1 - 1 = 0$. Meaning that we cannot satisfy the condition of the for-loop. 

Thus concluding that $n$ is prime with a probability of $1 - 4^{-k} = 1 - 4^{-3} = 1 - 0.015625 = 0.984375 \approx 98\%$

The same is attached in \verb|miller_rabin.sage|.
We have written a somewhat clunky method to factor $n-1$ into powers of two and an odd factor, but it works for the case $n=263$.

\begin{verbatim}
sage: load miller_rabin.sage
n-1 = 262 = 2^1 * 131 
prime with probability 0.98
\end{verbatim}

\section{Question 3: Pollard's Rho factorisation}

If we implement Pollard's Rho factorisation algorithm as attached (see \verb|pollard.sage|). \\
\noindent
We note that we get the following factors: $479951 \cdot 479939 \cdot 479909 = 110545695839248001$.



\section{Question 4: $p - 1$ factorisation}
We implemented the $p-1$ algorithm as attached in \verb|p-1.sage|.
\begin{verbatim}
sage: load p-1.sage
n = 53098980256925153592047 = 480061*110608818997846427
\end{verbatim}

\end{document}
