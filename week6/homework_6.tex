\documentclass{article}
\usepackage{amsfonts}
\usepackage{amsmath}
\usepackage{hyperref}
\usepackage[left=2cm,right=2cm,top=2cm,bottom=2cm]{geometry}
%\newenvironment{amatrix}[1]{%
%  \left(\begin{array}{@{}*{#1}{c}|c@{}}
%}{%
%  \end{array}\right)
%}

\author{Mark Vijfvinkel \& Aram Verstegen \\ 0863002(s134674), s4092368 \\ Radboud University}
\title{Cryptography 1, Homework 6}

\begin{document}
\maketitle
\date

\section{Non-field}
%We can read this question as asking us to find two elements 

\section{Prove interesting property}

To prove:
\[ (x+y)^{p^n} = x^{p^n} + y^{p^n} \]
Basis, $n=0$:
\[ (x+y)^{p^0} = (x+y)^1 = x + y \]

Induction, $n=n+1$:
\[ (x+y)^{p^n} = \sum_{k=0}^{p^n} {p^n \choose k} x^{p^n-k} y^{k}  \]
\[ = {p^n \choose 0} x^{p^n} y^{0} + {p^n \choose 1} x^{p^n-1} y^{1} + {p^n \choose 2} x^{p^n-2} y^{2}~ \dots~{p^n \choose p^n-2} x^{2} y^{p^n-2} + {p^n \choose p^n-1} x^{1} y^{p^n-1} + {p^n \choose p^n} x^{0} y^{p^n} \]
We note that the first and last terms are equal to $x^{p^n}$ and $y^{p^n}$ respectively, and every binomial coefficient in between is divisible by $p$.
We know that fields with prime characteristics are isomorphic to integers modulo that prime, and this information lets us reduce these terms all at once.
\[ = x^{p^n} + y^{p^n}~\square \]

%Basis, $n=1$:
%\[ (x+y)^p = \sum_{k=0}^{p} {p \choose k} x^{p-k} y^{k}  \]
%\[ = {p \choose 0} x^{p} y^{0} + {p \choose 1} x^{p-1} y^{1} + {p \choose 2} x^{p-2} y^{2}~ \dots~{p \choose p-2} x^{2} y^{p-2} + {p \choose p-1} x^{1} y^{p-1} + {p \choose p} x^{0} y^{p} \]
%We note that the first and last terms are equal to $x^p$ and $y^p$ respectively, and every binomial coefficient in between is divisible by $p$.
%We know that fields with prime characteristics are isomorphic to integers modulo that prime, and this information lets us reduce these terms all at once.
%\[ = x^{p} + y^{p} \pmod p. \]

%Induction, $n=n+1$:
%\begin{equation}
%\begin{aligned}
%& (x+y)^{p^{n+1}} \\ 
%&= (x+y)^{p} \cdot (x+y)^{p^{n}} \\
%&= (x^p+y^p) \cdot (x+y)^{p^{n}} \\
%&= x^p \cdot (x+y)^{p^{n}} + y^p \cdot (x+y)^{p^{n}} \\
%\end{aligned}
%\end{equation}


\end{document}
