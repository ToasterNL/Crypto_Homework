\documentclass{article}
\usepackage{amsfonts}
\usepackage{amsmath}
\usepackage{hyperref}
\usepackage[left=2cm,right=2cm,top=2cm,bottom=2cm]{geometry}
%\newenvironment{amatrix}[1]{%
%  \left(\begin{array}{@{}*{#1}{c}|c@{}}
%}{%
%  \end{array}\right)
%}

\author{Mark Vijfvinkel \& Aram Verstegen \\ 0863002(s134674), s4092368 \\ Radboud University}
\title{Cryptography 1, Homework 6}

\begin{document}
\maketitle
\date

\section{Non-field}
%We can read this question as asking us to find two elements
If we take $n=2$, we have the function $(x+1)^2$. This results in $x^2+2x+1$, which in $\mathbb{F}_2$ results in $x^2+2x+1 \mbox{ mod } 2 = x^2+1$.

If we put this in a table for multiplication we get the following:
\begin{tabular}{|c|c|c|c|c|}
\hline 
$\cdot$ & $0$ & $1$ & $x$ & $x+1$ \\ 
\hline 
$0$ & $0$ & $0$ & $0$ & $0$ \\ 
\hline 
$1$ & $0$ & $1$ & $x$ & $x+1$ \\ 
\hline 
$x$ & $0$ & $x$ & $1$ & $x+1$ \\ 
\hline 
$x+1$ & $0$ & $x+1$ & $x+1$ & $0$ \\ 
\hline 
\end{tabular}

This shows that we do get $0$ when we multiply a non-zero element with another non-zero element for $n=2$. Thus $\mathbb{R}$ is not a field.

\section{Prove interesting property}

To prove:
\[ (x+y)^{p^n} = x^{p^n} + y^{p^n} \]
Basis, $n=0$:
\[ (x+y)^{p^0} = (x+y)^1 = x + y = x^{p^0} + y^{p^0} \]

Basis, $n=1$:
\[ (x+y)^p = \sum_{k=0}^{p} {p \choose k} x^{p-k} y^{k}  \]
\[ = {p \choose 0} x^{p} y^{0} + {p \choose 1} x^{p-1} y^{1} + {p \choose 2} x^{p-2} y^{2}~ \dots~{p \choose p-2} x^{2} y^{p-2} + {p \choose p-1} x^{1} y^{p-1} + {p \choose p} x^{0} y^{p} \]
We note that the first and last terms are equal to $x^p$ and $y^p$ respectively, and every binomial coefficient in between is divisible by $p$, which means that these middle terms go to zero modulo $p$.
\[ = x^{p} + y^{p} \pmod p. \]

%Induction, $n=n+1$:
%\begin{equation}
%\begin{aligned}
%& (x+y)^{p^{n+1}} \\ 
%&= (x+y)^{p} \cdot (x+y)^{p^{n}} \\
%&= (x^p+y^p) \cdot (x+y)^{p^{n}} \\
%&= x^p \cdot (x+y)^{p^{n}} + y^p \cdot (x+y)^{p^{n}} \\
%\end{aligned}
%\end{equation}

Induction, $n=n+1$:
\begin{equation*}
\begin{aligned}
(x+y)^{p^{n+1}} &= (x+y)^p \cdot (x+y)^{p^n} \\
&= (x^p+y^p) \cdot (x^{p^n}+y^{p^n}) \\
\end{aligned}
\end{equation*}
We feel like we may be skipping a step here, but know that in fields we can multiply polynomials like vectors:
\begin{equation*}
\begin{aligned}
&= x^p x^{p^n} + y^p y^{p^n} \\
&= x^{p^{n+1}} + y^{p^{n+1}}~\square.
\end{aligned}
\end{equation*}

\section{Compute $N_3(4)$} 

We have a .pdf online that makes this problem very easy to compute.
\footnote{\url{http://www.math.umn.edu/~garrett/coding/Overheads/21_counting.pdf}}

The following formula is used:

\[
\frac{p^{q^2} - p^q}{q^2}
\]

where $p$ is the field, in this case $3$, $q$ is the degree squared in this case $2$.
This result in the following equation:

\[
\frac{3^{2^2} - 3^2}{2^2} = \frac{3^4 - 3^2}{4} = 18
\]


\end{document}
