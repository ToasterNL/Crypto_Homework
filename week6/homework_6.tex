\documentclass{article}
\usepackage{amsfonts}
\usepackage{amsmath}
\usepackage{hyperref}
\usepackage[left=2cm,right=2cm,top=2cm,bottom=2cm]{geometry}
%\newenvironment{amatrix}[1]{%
%  \left(\begin{array}{@{}*{#1}{c}|c@{}}
%}{%
%  \end{array}\right)
%}

\author{Mark Vijfvinkel \& Aram Verstegen \\ 0863002(s134674), s4092368 \\ Radboud University}
\title{Cryptography 1, Homework 6}

\begin{document}
\maketitle
\date

\section{Non-field}
We can read this question as asking us to find two nonzero elements that result in zero when multiplied.
We weren't sure if this question is posed in concreto or in abstracto.
Let's start with a concrete example:

We choose $n=2$, and draw a table for multiplication:

\begin{tabular}{|c|c|c|c|}
\hline 
$\cdot$ & $1$ & $x$ & $x+1$ \\ 
\hline 
$1$ & $1$ & $x$ & $x+1$ \\ 
\hline 
$x$ & $x$ & $1$ & $x+1$ \\ 
\hline 
$x+1$ & $x+1$ & $x+1$ & $0$ \\ 
\hline 
\end{tabular}

For $(x+1)^2$ we find $x^2+2x+1$, which in $\mathbb{F}_2$ results in $x^2+1 = 0 \pmod{x^2+1} $.
This shows that we \emph{do} get $0$ when we multiply a non-zero element with another non-zero element for $n=2$. Thus $\mathbb{R}$ is not a field.

More generally, we can expand this for all $n$ as follows:

$(x+1)^n = (x^n+1) = (x+1)^{(n-2)} \cdot (x+1)^2 $, since we already established that $(x+1)^2$ results in 0, everything will result in 0. Therefore $\mathbb{R}$ is not a field for all $n$.

Even more generally, we can always construct a number $z = x^{n-1} + \frac{1}{x}$ so that $zx = x^n + 1 = 0 \pmod{x^n+1}$.
Since we are working with a (non-)field of characteristic $2$, $\frac{1}{x}$ should then be interpreted as the multiplicative inverse of $x$ modulo $2$.
To show multiplicative inverses for $x$ always exist we need to show that it is coprime to the field characteristic 2, so $gcd(x,2) = 1$, which follows from the primality of 2.
Every $x$ is coprime to 2 and thus an inverse exists, and thus we can always find an element $z$ that gives zero when multiplied by $x$. This element $z$ is always non-zero because it can be constructed within the (non-)field, and thus $\mathbb{R}$ is never a field because we can construct zero by multiplying two non-zero values within it.

\section{Prove interesting property}

We already can see that this is true, because $p^n$ is the order of the group, and exponentiation by the group order works like the identity function.
To prove:
\[ (x+y)^{p^n} = x^{p^n} + y^{p^n} \]
Basis, $n=0$:
\[ (x+y)^{p^0} = (x+y)^1 = x + y = x^{p^0} + y^{p^0}. \]

Basis, $n=1$:
\[ (x+y)^p = \sum_{k=0}^{p} {p \choose k} x^{p-k} y^{k}  \]
\[ = {p \choose 0} x^{p} y^{0} + {p \choose 1} x^{p-1} y^{1} + {p \choose 2} x^{p-2} y^{2}~ \dots~{p \choose p-2} x^{2} y^{p-2} + {p \choose p-1} x^{1} y^{p-1} + {p \choose p} x^{0} y^{p} \]
We note that the first and last terms are equal to $x^p$ and $y^p$ respectively, and every binomial coefficient in between is divisible by $p$, which means that these middle terms go to zero modulo $p$.
\[ = x^{p} + y^{p} \pmod p. \]

%Induction, $n=n+1$:
%\begin{equation}
%\begin{aligned}
%& (x+y)^{p^{n+1}} \\ 
%&= (x+y)^{p} \cdot (x+y)^{p^{n}} \\
%&= (x^p+y^p) \cdot (x+y)^{p^{n}} \\
%&= x^p \cdot (x+y)^{p^{n}} + y^p \cdot (x+y)^{p^{n}} \\
%\end{aligned}
%\end{equation}

Induction, $n=n+1$:
\begin{equation*}
\begin{aligned}
%(x+y)^{p^{n+1}} &= x^{p^{n+1}}+y^{p^{n+1}} \\
%(x+y)^p (x+y)^{p^{n}} &= x^{p^{n+1}}+y^{p^{n+1}} \\
%(x^p+y^p) (x^{p^{n}}+y^{p^{n}}) &= x^{p^{n+1}}+y^{p^{n+1}}~(IH) \\
%\text{Applied Fermat's little theorem:} \\
%(x+y) (x^{p^{n}}+y^{p^{n}}) &= x^{p^{n+1}}+y^{p^{n+1}}~\pmod{p} \\
%\text{Applied Frobenius automorphism:} \\
%(x+y) (x+y) &= x^{1+1}+y^{1+1}~\pmod{p} \\
%(x+y)^2 &= x^{2}+y^{2}~\pmod{p} \\
%--
%(x+y)^{p^{n}} &= (x+y)^{p^n} \\
%&= (x+y)^p \cdot (x+y) \\
%&= (x+y)^{p+1} \\
(x+y)^{p^{n}} &= (x^{p^n}+y^{p^n})~\text{(IH)} \\
((x+y)^{p^{n}})^p &= (x^{p^n}+y^{p^n})^p \\
((x+y)^{p^{n}})^p &= (x^{p^n})^p+(y^{p^n})^p~\text{(Basis)}\\
(x+y)^{p^{n+1}} &= x^{p^{n+1}}+y^{p^{n+1}}.~\square\\
%&= x^{p^{n+1}}+y^p \cdot (x^{p^n}+y^{p^n}) \\
%&= (x+y)^p \cdot (x^{p^n}+y^{p^n}) \\
%&= (x^{p+1}+y^{p+1})~(IH) \\
%&= (x^{p+1}+y^{p+1})^{p^n} \\
%&= (x^{p^{n+1}+p^n}+y^{p^{n+1}+p^n}) \\
%(x+y)^{p^{n+1}} &= x^{p^{n+1}}+y^{p^{n+1}} + y^p x^{p^n} + x^p y^{p^n} \\
%(x+y)^{p^{n+1}} &= x^{p^{n+1}}+y^{p^{n+1}} + y^p x^{p^n} + x^p y^{p^n} \\
%&= x^{p^{n+1}}+y^{p^{n+1}} + y^p x + x^p y \\
%&= (x^p+y^p) \cdot (x^{p^n}+y^{p^n}) \\
%&= x^p x^{p^n} + y^p x^{p^n} + x^p y^{p^n} + y^p y^{p^n} \\
%&= x^p x + y^p x + x^p y + y^p y \\
%&= x(x^{p} + y^p) + y (x^p + y^p) \\
%&= x(x+ y)^p + y (x+y)^p \\
%&= (x+y)(x+ y)^p\\
%&= x^p x^{p^n} + y^p y^{p^n} \\
%&= x^{p^{n+1}} + y^{p^{n+1}}.~\square
%%\end{aligned}
%%\end{equation*}
%%We feel like we may be skipping a step here, but know that in fields we can multiply polynomials like vectors:
%%\begin{equation*}
%%\begin{aligned}
\end{aligned}
\end{equation*}

\section{Compute $N_3(4)$} 

We have found a .pdf online that makes this problem very easy to compute.
\footnote{\url{http://www.math.umn.edu/~garrett/coding/Overheads/21_counting.pdf}}

The following formula is used:

\[
\frac{p^{q^2} - p^q}{q^2}
\]

where $p$ is the field, in this case $3$, $q$ is the degree squared in this case $2$.
This results in the following equation:

\[
\frac{3^{2^2} - 3^2}{2^2} = \frac{3^4 - 3^2}{4} = 18
\]

\section{Rabin test}
\begin{verbatim}
sage: x=ZZ['x'].0
sage: f=x**121 + x**2 + 1
sage: factor(121)
11^2
sage: h=(x^(2^11)-x) % f
sage: gcd(f,h)
1
sage: root=x^(2^11)
sage: ((root % f) * (root % f) - x) % f == 0
False
\end{verbatim}
We can see that both factors of the leading term of $f$ are the same, so we need to do only one test with gcd on a divisor.
We get around exponents of degree $2^{121}$ by working modulo $f$.
We see that the polynomial is reducible.

%\section{Rabin Test}
%Input is $f(x) = x^{121}+x^2+1 \in \mathbb{F}_2^{[x]}$, $deg(f) = 121$.
%We first determine the prime factors of 121. This is not hard, $11*11=121$.
%The first step is to see if $f(x)$ divides $x^{2^{121}}-x$

\end{document}
