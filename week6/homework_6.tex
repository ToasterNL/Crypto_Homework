\documentclass{article}
\usepackage{amsfonts}
\usepackage{amsmath}
\usepackage{hyperref}
\usepackage[left=2cm,right=2cm,top=2cm,bottom=2cm]{geometry}
%\newenvironment{amatrix}[1]{%
%  \left(\begin{array}{@{}*{#1}{c}|c@{}}
%}{%
%  \end{array}\right)
%}

\author{Mark Vijfvinkel \& Aram Verstegen \\ 0863002(s134674), s4092368 \\ Radboud University}
\title{Cryptography 1, Homework 6}

\begin{document}
\maketitle
\date

\section{Non-field}
We can read this question as asking us to find two nonzero elements that result in zero when multiplied.
We weren't sure if this question is posed in concreto or in abstracto.
Let's start with a concrete example:

We choose $n=2$, and draw a table for multiplication:

\begin{tabular}{|c|c|c|c|}
\hline 
$\cdot$ & $1$ & $x$ & $x+1$ \\ 
\hline 
$1$ & $1$ & $x$ & $x+1$ \\ 
\hline 
$x$ & $x$ & $1$ & $x+1$ \\ 
\hline 
$x+1$ & $x+1$ & $x+1$ & $0$ \\ 
\hline 
\end{tabular}

For $(x+1)^2$ we find $x^2+2x+1$, which in $\mathbb{F}_2$ results in $x^2+1 = 0 \pmod{x^2+1} $.
This shows that we \emph{do} get $0$ when we multiply a non-zero element with another non-zero element for $n=2$. Thus $\mathbb{R}$ is not a field.

More generally, we can expand this for all $n$ as follows:

$(x+1)^n = (x^n+1) = (x+1)^{(n-2)} \cdot (x+1)^2 $, since we already established that $(x+1)^2$ results in 0, everything will result in 0. Therefore $\mathbb{R}$ is not a field for all $n$.

Even more generally, we can always construct a number $n = x^{n-1} + \frac{1}{x}$ so that $nx = x^n + 1 = 0 \pmod{x^n+1}$.
Since we are working in a group, $\frac{1}{x}$ should then be interpreted as the multiplicative inverse of $x$ modulo $x^n+1$.
To show multiplicative inverses for $x$ always exist we need to show that $gcd(x,x^n+1) = 1$, which follows from $x^n = 0 \pmod x$, so the GCD is 1 and thus an inverse exists, and thus we can always find a non-zero element that gives zero when multiplied by $x$, and thus $\mathbb{R}$ is not a field.

\section{Prove interesting property}

To prove:
\[ (x+y)^{p^n} = x^{p^n} + y^{p^n} \]
Basis, $n=0$:
\[ (x+y)^{p^0} = (x+y)^1 = x + y = x^{p^0} + y^{p^0} \]

Basis, $n=1$:
\[ (x+y)^p = \sum_{k=0}^{p} {p \choose k} x^{p-k} y^{k}  \]
\[ = {p \choose 0} x^{p} y^{0} + {p \choose 1} x^{p-1} y^{1} + {p \choose 2} x^{p-2} y^{2}~ \dots~{p \choose p-2} x^{2} y^{p-2} + {p \choose p-1} x^{1} y^{p-1} + {p \choose p} x^{0} y^{p} \]
We note that the first and last terms are equal to $x^p$ and $y^p$ respectively, and every binomial coefficient in between is divisible by $p$, which means that these middle terms go to zero modulo $p$.
\[ = x^{p} + y^{p} \pmod p. \]

%Induction, $n=n+1$:
%\begin{equation}
%\begin{aligned}
%& (x+y)^{p^{n+1}} \\ 
%&= (x+y)^{p} \cdot (x+y)^{p^{n}} \\
%&= (x^p+y^p) \cdot (x+y)^{p^{n}} \\
%&= x^p \cdot (x+y)^{p^{n}} + y^p \cdot (x+y)^{p^{n}} \\
%\end{aligned}
%\end{equation}

Induction, $n=n+1$:
\begin{equation*}
\begin{aligned}
(x+y)^{p^{n+1}} &= (x+y)^p \cdot (x+y)^{p^n} \\
&= (x^p+y^p) \cdot (x^{p^n}+y^{p^n}) \\
\end{aligned}
\end{equation*}
We feel like we may be skipping a step here, but know that in fields we can multiply polynomials like vectors:
\begin{equation*}
\begin{aligned}
&= x^p x^{p^n} + y^p y^{p^n} \\
&= x^{p^{n+1}} + y^{p^{n+1}}~\square.
\end{aligned}
\end{equation*}

\section{Compute $N_3(4)$} 

We have found a .pdf online that makes this problem very easy to compute.
\footnote{\url{http://www.math.umn.edu/~garrett/coding/Overheads/21_counting.pdf}}

The following formula is used:

\[
\frac{p^{q^2} - p^q}{q^2}
\]

where $p$ is the field, in this case $3$, $q$ is the degree squared in this case $2$.
This result in the following equation:

\[
\frac{3^{2^2} - 3^2}{2^2} = \frac{3^4 - 3^2}{4} = 18
\]


\end{document}
