\documentclass{article}
\usepackage{amsfonts}
\usepackage{amsmath}
\usepackage{hyperref}
\usepackage[left=2cm,right=2cm,top=2cm,bottom=2cm]{geometry}
%\newenvironment{amatrix}[1]{%
%  \left(\begin{array}{@{}*{#1}{c}|c@{}}
%}{%
%  \end{array}\right)
%}

\author{Mark Vijfvinkel \& Aram Verstegen \\ 0863002(s134674), 100863(s4092368) \\ Radboud University}
\title{Cryptography 1, Homework 12}

\begin{document}
\maketitle
\date

\section{Question 1 \& 2:}
Sage has a very nice function named: euler\_phi(n).
\\
We used this to calculate question 1 and 2, the results are shown below: 
\\
Question 1: 8640
\\
We could also use factor in sage to obtain: $2^3*3^3*5^2*7$
\\
Then we use the product formula: $37800*(1-(1/2))*(1-(1/3))*(1-(1/5))*(1-(1/7)) = 8640$
\\
Question 2: 379247933987370471260160
\\
We could also use factor in sage to obtain: $2^17 * 3^12 * 5 * 7^5 * 11^7 * 17$
\\
Then we use the product formula: 
\\
$1939201349958859167498240*(1-(1/2))*(1-(1/3))*(1-(1/5))*(1-(1/7))*(1-(1/11))*(1-(1/17)) = 379247933987370471260160$

\section{Question 3:}
The public key (e, n): (23441, $p \cdot q$) $=$ (23441, 103487)
\\
The private key (n, d): (103487, $d \equiv e^{-1} \pmod{\phi(n)} $) $=$ (103487, $d \equiv 23441^{-1} \pmod{\phi(103487)} $) $=$
\\
(103487, $d \equiv 23441^{-1} \pmod{102816} $) $=$ (103487, 67889)

\section{Question 4:}

%TODO

\section{Question 5:}

We have the following congruences:
\\
\noindent
$x \equiv 0 \pmod 3$ \\
$x \equiv 1 \pmod 5$ \\
$x \equiv 2 \pmod 8$ \\

\noindent
We can transform them in the following system of equations:
\\
\noindent
$ x = 0 + 3t$ \\
$ x = 1 + 5u$ \\
$ x = 2 + 8v$ \\

\noindent
We plug the first equation into the second congruence:
\\
\noindent
$0 + 3t \equiv 1 \pmod 5$ \\
$3t \equiv 1 \pmod 5$ \\
$t \equiv \frac{1}{3} \pmod 5$ \\
$t \equiv 2 \pmod 5$ \\
$t = 2 + 5u$\\

\noindent
We can plug this into the first equation:
\\
\noindent
$ x = 0 + 3t = 0 + 3 \cdot (2 + 5u) = 6 + 15u $ \\

\noindent
This we can plug into the third congruence:\\
\noindent
$6 + 15u \equiv 2 \pmod 8$ \\
$15u \equiv -4 \pmod 8$ \\
$u \equiv \frac{-4}{15} \pmod 8$ \\
$u \equiv 4 \pmod 8$ \\
$u = 4 + 8v$ \\

\noindent
We can plug this into the equation above: \\

$6 + 15 \cdot 4 + 8v = 6 + 60 \cdot 120v = 66 + 120v$\\

\noindent
So the smallest positive integer to satisfy the system of congruences is 66. The following values would be 66 plus a multiple of a 120.

\section{Question 6:}

We know that $m' = (m || 00...0) \oplus G(r)$ and $r' = r \oplus H(m')$
\\
First we have to recover the random string $r = r \oplus H(m')$
\\
Then, to recover the message $m || 00...0 = m' \oplus G(r)$
\\
The result is the message concatenated with the padding.


\end{document}
