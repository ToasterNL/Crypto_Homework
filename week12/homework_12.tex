\documentclass{article}
\usepackage{amsfonts}
\usepackage{amsmath}
\usepackage{hyperref}
\usepackage[left=2cm,right=2cm,top=2cm,bottom=2cm]{geometry}
%\newenvironment{amatrix}[1]{%
%  \left(\begin{array}{@{}*{#1}{c}|c@{}}
%}{%
%  \end{array}\right)
%}

\author{Mark Vijfvinkel \& Aram Verstegen \\ 0863002(s134674), 100863(s4092368) \\ Radboud University}
\title{Cryptography 1, Homework 12}

\begin{document}
\maketitle
\date

\section{}
Sage has a nice function named: $euler\_phi(n)$, which we will not use here.
We factor the number by hand, starting from the fact that the number is a multiple of 100.
From then on it's easy to divide the remainder by two or three a few times until we reach a factor of $7$.
We could also use $factor()$ in sage to obtain: $2^3 \cdot 3^3 \cdot 5^2 \cdot 7$.

$\phi(37800)$ is then: $37800\cdot(1-\frac{1}{2}) \cdot (1-\frac{1}{3}) \cdot (1-\frac{1}{5}) \cdot (1-\frac{1}{7})$ = 8640.

%\\
%We used this to calculate question 1 and 2, the results are shown below: 
%\\
%Question 1: 8640
%\\
%We could also use factor in sage to obtain: $2^3*3^3*5^2*7$
%\\
%Then we use the product formula: $37800*(1-(1/2))*(1-(1/3))*(1-(1/5))*(1-(1/7)) = 8640$
%\\
\section{}
We first notice the number is even. By printing it in binary we can easily see how many 2-powers are in it.
\begin{verbatim}
0b110011010101001000101000011011001111100100111111000010010000100100000000000000000
\end{verbatim}

That's a string of 17 zeroes at the end, so we know that $2^{17}$ is a divisor.
After dividing by $2^{17}$, we end up with $14794932174368737545$ which is clearly a multiple of $5$.
Then we get a few divisors that are multiples of 3.

At this point the remaining number is only 43 bits long.
We can start Pollard's Rho or, in this case, we plugged it in the UNIX `factor' utility.
We could also use $factor()$ in sage to obtain: $2^{17} \cdot 3^{12} \cdot 5 \cdot 7^5 \cdot 11^7 \cdot 17$

Then we use the totient formula: 
$\phi(1939201349958859167498240) = 1939201349958859167498240 \cdot (1-(1/2)) \cdot (1-(1/3)) \cdot (1-(1/5)) \cdot (1-(1/7)) \cdot (1-(1/11)) \cdot (1-(1/17)) = 379247933987370471260160$

\section{}
The public key $(n, e)$: $(p \cdot q, 23441) = (103487, 23441)$.
\\
The private key $(n, d)$: $(103487, d)$.

$d \equiv e^{-1} \equiv 23441^{-1}$. % \pmod{\phi(103487)} ) = (103487, d \equiv 23441^{-1} \pmod{102816} ) = (103487, 67889)$.

$\phi(103487) = 102816 $. \\
Sanity check:
\begin{verbatim}
sage: pow(23441,(102816),103487)
1
\end{verbatim}
$d \equiv e^{-1} \equiv e^{\phi(n)-1}  \equiv 48907 \pmod n$

Sanity check \#2:
\begin{verbatim}
sage: 23441*48907 % 103487
1
\end{verbatim}

The private key $(n, d)$: $(103487, 48907)$.

\section{}
The message we will send is: $m^e \pmod n$, so $23^{17} \mod 11584115749$.
We can use the square and multiply method, multiplying $23$ with itself until we reach a wraparound at the modulus.
At $23^8$ we wrap around, and continue modulo $n$ with $23^8 \mod 11584115749 = 8806290787$ which we continue to multiply by $23$ and taking the modulus at any or every step.

Eventually we compute $23^{17} \mod 11584115749 = 10912105332$.

%Incidentally, their private part of the key, $d$, is $1362837147$.
%We can use this confirm we did it correctly; 

\section{}

We have the following congruences:
\\
\noindent
$x \equiv 0 \pmod 3$ \\
$x \equiv 1 \pmod 5$ \\
$x \equiv 2 \pmod 8$ \\

\noindent
We can transform them in the following system of equations:
\\
\noindent
$ x = 0 + 3t$ \\
$ x = 1 + 5u$ \\
$ x = 2 + 8v$ \\

\noindent
We plug the first equation into the second congruence:
\\
\noindent
$0 + 3t \equiv 1 \pmod 5$ \\
$3t \equiv 1 \pmod 5$ \\
$t \equiv \frac{1}{3} \pmod 5$ \\
$t \equiv 2 \pmod 5$ \\
$t = 2 + 5u$\\

\noindent
We can plug this into the first equation:
\\
\noindent
$ x = 0 + 3t = 0 + 3 \cdot (2 + 5u) = 6 + 15u $ \\

\noindent
This we can plug into the third congruence:\\
\noindent
$6 + 15u \equiv 2 \pmod 8$ \\
$15u \equiv -4 \pmod 8$ \\
$u \equiv \frac{-4}{15} \pmod 8$ \\
$u \equiv 4 \pmod 8$ \\
$u = 4 + 8v$ \\

\noindent
We can plug this into the equation above: \\

$6 + 15 \cdot 4 + 8v = 6 + 60 \cdot 120v = 66 + 120v$\\

\noindent
So the smallest positive integer to satisfy the system of congruences is 66. The following values would be 66 plus a multiple of a 120.

\section{Question 6:}

We know that $m' = (m || 00...0) \oplus G(r)$ and $r' = r \oplus H(m')$
\\
First we have to recover the random string $r = r \oplus H(m')$
\\
Then, to recover the message $m || 00...0 = m' \oplus G(r)$
\\
The result is the message concatenated with the padding.


\end{document}
