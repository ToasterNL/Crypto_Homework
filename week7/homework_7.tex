\documentclass{article}
\usepackage{amsfonts}
\usepackage{amsmath}
\usepackage{hyperref}
\usepackage[left=2cm,right=2cm,top=2cm,bottom=2cm]{geometry}
%\newenvironment{amatrix}[1]{%
%  \left(\begin{array}{@{}*{#1}{c}|c@{}}
%}{%
%  \end{array}\right)
%}

\author{Mark Vijfvinkel \& Aram Verstegen \\ 0863002(s134674), s4092368 \\ Radboud University}
\title{Cryptography 1, Homework 7}

\begin{document}
\maketitle
\date

\section{Question 1}

$m = 42$\\
$h_{A} = 224$\\
$k = 654$\\
$g = 3$\\
$\hbox{$c_1$} = 3^{654} \pmod{1013} = 628$\\
$\hbox{$s$} = h_A^k = 224^{654} \pmod{1013} = 1004 $\\
$\hbox{$c_2$} = 42 \cdot 1004 \pmod{1013} = 635$

The message sent to Alice will then be $(628, 635)$

If we look ahead to question 3, we calculated that Alice has a secret random value of 612. We can calculate the decryption to check the answer:\\

$628^{612} \pmod{1013} = 1004$\\
$\hbox{Multiplicative inverse} = 1004^{-1} \pmod{1013} = 225$\\
$\hbox{Decryption:} 225 \cdot 635 \pmod{1013} = 42$

\section{Question 2}
The product of all monic, irreducible polynomials of degree 6 over $\mathbb{F}_2$.

\[
\prod_{d|n}(x^{q^{\frac{n}{d}}}-x)^{\mu(d)}
=
(x^{2^{6}}-x)^{\mu(1)}
(x^{2^{3}}-x)^{\mu(2)}
(x^{2^{2}}-x)^{\mu(3)}
(x^{2^{1}}-x)^{\mu(6)}
\]
\[
=
(x^{64}-x)^{\mu(1)}
(x^{8}-x)^{\mu(2)}
(x^{4}-x)^{\mu(3)}
(x^{2}-x)^{\mu(6)}
\]
\[
=
(x^{64}-x)^{1}
(x^{8}-x)^{-1}
(x^{4}-x)^{-1}
(x^{2}-x)^{1}
\]
\[
=
\frac{
(x^{64}-x)
(x^{2}-x)
}{
(x^{8}-x)
(x^{4}-x)
}
\]
\[
=
\frac{
x^{66}-x^3-x^{65}+x^2
}{
x^{12}-x^5+x^9+x^2
}
\]
\[
=
\frac{
x^{66}+x^{65}+x^3+x^2
}{
x^{12}+x^9+x^5+x^2
}
\]

\section{Question 3}

$\hbox{Group order }l = 1012$\\
$m = \lfloor \sqrt{l} \rfloor = \lfloor \sqrt{1012} \rfloor = 31$\\
$3^{-1} \pmod{1013} = 338$\\
$d = 338^{31} \pmod{1013} = 406$\\

We will first describe the baby-steps, $3^i \pmod{1013}$:\\

\begin{tabular}{|c|c|c|c|c|c|c|c|c|c|c|c|c|c|c|c|c|}
\hline 
$i = $ & 0 & 1 & 2 & 3 & 4 & 5 & 6 & 7 & 8 & 9 & 10 & 11 & 12 & 13 & 14 & 15 \\ 
\hline 
• & 1 & 3 & 9 & 27 & 81 & 243 & 729 & 161 & 483 & 436 & 295 & 885 & 629 & 874 & 596 & 775 \\ 
\hline 
$i = $ & 16 & 17 & 18 & 19 & 20 & 21 & 22 & 23 & 24 & 25 & 26 & 27 & 28 & 29 & 30 & 31 \\ 
\hline 
• & 299 & 897 & 665 & 982 & 920 & 734 & 176 & 528 & 571 & 700 & 74 & 222 & 666 & 985 & 929 & 761 \\ 
\hline
$i = $ & 31 & & & & & & & & & & & & & & & \\
\hline
 	   & 257& & & & & & & & & & & & & & & \\
 \hline
\end{tabular}\\

We will describe the big-steps, $224 \cdot 406^j \pmod{1013}$, until we reach a value that matches with the baby-steps:\\

\begin{tabular}{|c|c|c|c|c|c|c|c|c|c|c|c|c|c|c|c|c|c|c|c|c|}
\hline 
$i = $ & 0 & 1 & 2 & 3 & 4 & 5 & 6 & 7 & 8 & 9 & 10 & 11 & 12 & 13 \\ 
\hline 
• & 224 & 787 & 427 & 139 & 719 & 170 & 136 & 514 & 6 & 410 & 328 & 465 & 372 & 95 \\ 
\hline 
$i = $ & 14 & 15 & 16 & 17 & 18 & 19 & & & & & & & & \\
\hline
 & 76 & 466 & 778 & 825 & 660 & 528 & & & & & & & & \\
\hline
\end{tabular}\\
\\
To calculate the random value, $31 \cdot 19 + 23 = 612$. \\
To check this value $3^{612} \pmod{1013} = 224$

\section{Question 4}

$\hbox{Group order} = 1012 = 4 \cdot 11 \cdot 23$\\
$g = 3$\\
$h = 321$\\
We have to calculate $x$ in $321 \equiv 3^x \pmod{1013}$\\

The multiplicative inverse is, $3^{-1} mod 1013 = 338$.\\
We first calculate the roots:\\
$3^{1012/4} \pmod{1013} = 3^{253} \pmod{1013} = 968$\\
$3^{1012/11} \pmod{1013} = 3^{92} \pmod{1013} = 122$\\
$3^{1012/23} \pmod{1013} = 3^{44} \pmod{1013} = 586$\\

We now compute the tables for $p_{1} = 4, p_{2} = 11, p_{3} = 23$: \\

\begin{tabular}{|c|c|c|c|c|}
\hline 
 $p_{1} = 4$ & 0 & 1 & 2 & 3 \\ 
\hline 
• & 1 & 986 & 1012 & 45 \\ 
\hline 
\end{tabular}\\ 

\begin{tabular}{|c|c|c|c|c|c|c|c|c|c|c|c|}
\hline 
$p_{2} = 11$ & 0 & 1 & 2 & 3 & 4 & 5 & 6 & 7 & 8 & 9 & 10 \\ 
\hline 
• & 1 & 122 & 702 & 552 & 486 & 538 & 804 & 840 & 167 & 114 & 739 \\ 
\hline 
\end{tabular}\\

\begin{tabular}{|c|c|c|c|c|c|c|c|c|c|c|c|c|}
\hline 
$p_{3} = 23$ & 0 & 1 & 2 & 3 & 4 & 5 & 6 & 7 & 8 & 9 & 10 & 11 \\ 
\hline 
• & 1 & 586 & 1002 & 645 & 121 & 1009 & 695 & 44 & 459 & 529 & 16 & 259 \\ 
\hline 
• & 12 & 13 & 14 & 15 & 16 & 17 & 18 & 19 & 20 & 21 & 22 & • \\ 
\hline 
• & 837 & 190 & 923 & 949 & 990 & 704 & 253 & 360 & 256 & 92 & 223 & • \\ 
\hline 
\end{tabular}

\noindent
We can now calculate the following values:\\
$321^{1012/4} \pmod 1013 = 1012$, looking this value up in table $p_{1} = 4$ gives an index of 2\\
$321^{1012/11} \pmod 1013 = 804$, looking this value up in table $p_{2} = 11$ gives an index of 6\\
$321^{1012/23} \pmod 1013 = 190$, looking this value up in table $p_{3} = 23$ gives an index of 13\\

The next step is to determine the values of $u, v, w$, where:

\begin{tabular}{ccc}
$u \equiv 1 \pmod{4}$ & $u \equiv 0 \pmod{11}$ & $u \equiv 0 \pmod{23}$ \\ 
%\hline 
$v \equiv 0 \pmod{4}$ & $v \equiv 1 \pmod{11}$ & $u \equiv 0 \pmod{23}$ \\ 
%\hline 
$w \equiv 0 \pmod{4}$ & $w \equiv 0 \pmod{11}$ & $w \equiv 1 \pmod{23}$ \\ 
\end{tabular} \\

For $u$ this means: $u = k \cdot 11 \cdot 23 = k \cdot 253$ and $u = 4\cdot l + 1$.\\
Thus $253k = 4l+1 = 253k - 4l = 1$\\
We can solve this using the extended Euclidean algorithm:

$253 = 253 \cdot 1 + 4 \cdot 0$

$4 = 253 \cdot 0 + 4 \cdot 1$

$1 = 253 \cdot 1 - 63 \cdot 4$, because $253 = 63 \cdot 4 + 1$

$u = 4 \cdot 63 + 1$

$u = 1 \cdot 253$

We can fill $253$ in for the values of $u$ in the matrix and see that this is correct.\\

For $v$ this means: $v = k \cdot 4 \cdot 23 = k \cdot 92$ and $v = 11\cdot l + 1$.\\
Thus $92k = 11l+1 = 92k - 11l = 1$\\
We can solve this using the extended Euclidean algorithm:

$92 = 92 \cdot 1 + 11 \cdot 0$

$11 = 92 \cdot 0 + 11 \cdot 1$

$4 = 92 \cdot 1 - 8 \cdot 11$

$3 = (92 \cdot 0 + 11 \cdot 1) - 2 \cdot 92 \cdot 1 - 8 \cdot 11 = 92 \cdot -2 + 17 \cdot 11$

$1 = (92 \cdot 1 - 8 \cdot 11) - (92 \cdot -2 + 17 \cdot 11) = 92 \cdot 3 - 25 \cdot 11$

$v = 11 \cdot 25 + 1 = 276$

$v = 3 \cdot 92 = 276$

We can fill $276$ in for the values of $v$ in the matrix and see that this is correct.\\

For $w$ this means: $v = k \cdot 4 \cdot 11 = k \cdot 44$ and $v = 23\cdot l + 1$.\\
Thus $44k = 23l+1 = 44k - 23l = 1$\\
We can solve this using the extended Euclidean algorithm:

$44 = 44 \cdot 1 + 23 \cdot 0$

$23 = 44 \cdot 0 + 23 \cdot 1$

$21 = 44 \cdot 1 - 23 \cdot 1$

$2 = (44 \cdot 0 + 23 \cdot 1) - (44 \cdot 1 - 23 \cdot 1) = 44 \cdot -1 + 23 \cdot 2 $

$1 = (44 \cdot 1 - 23 \cdot 1) - 10 \cdot (44 \cdot -1 + 23 \cdot 2) = 44 \cdot 11 - 23 \cdot 21$

$w = 23 \cdot 21 + 1 = 484$

$w = 11 \cdot 4 \cdot 11 = 484$

We can fill $484$ in for the values of $v$ in the matrix and see that this is correct.\\

The secret value is: $p_{{1}_{i}} \cdot u + p_{{2}_{i}} \cdot v + p_{{3}_{i}} \cdot w$\\
We fill in the values: $ 2 \cdot 253 + 6 \cdot 276 + 13 \cdot 484 \pmod{1012} = 8454 \pmod{1012} = 358$

\section{Question 5}
\subsection{}
We need to show that $R = R'$, i.e. $g^k=g^{u_1} \cdot h_A^{u_2}$
\[ g^k=g^{u_1} \cdot h_A^{u_2} \]
%\[ g^k=g^{u_1} \cdot g^{a^{u_2}} \]
%\[ g^k=g^{u_1+a^{u_2}} \]
%\[ g^k=g^{u_1+a^{u_2}} \]
%\[ g^k=g^{(h(m) \cdot w \mod l)+a^{u_2}} \]
%\[ g^k=g^{(h(m) \cdot w \mod l)+a^{(r \cdot w~mod~l)}} \]
%\[ g^k=g^{(h(m) \cdot w \mod l)} \cdot g^{a^{(r \cdot w~mod~l)}} \]
\[ g^k=g^{(h(m) \cdot w \mod l)} \cdot h_A^{(r \cdot w~mod~l)} \]
%\[ h(m) = ar+sk \pmod l \]

%\[ g^k=g^{(h(m) \cdot w \mod l) + {a^{(r \cdot w~mod~l)}}} \]
%\[ g^{sk}=g^{s(h(m) \cdot w \mod l) + {a^{(r \cdot w~mod~l)}}} \]
\[ g^{sk}=g^{h(m)} \cdot h_A^{s(r \cdot w~mod~l)} \]
\[ g^{sk}=g^{h(m)} \cdot h_A^{r} \]

And we get stuck fairly early on.

%\[ s=k^{-1}(H(m)+r \cdot a) mod~l \]
%\[ w=s^{-1}=(k^{-1}(H(m)+r \cdot a))^{-1} mod~l \]

%\[ g^k=g^{(h(m) \cdot (k^{-1} \cdot (H(m)+r \cdot a))^{-1} \mod l)} \cdot h_A^{(r \cdot ((k^{-1} \cdot (H(m)+r \cdot a))^{-1})~mod~l)} \]

%\[ g^k=g^{(h(m) \cdot (k^{-1} \cdot (H(m)+r \cdot a))^{-1} \mod l)} \cdot h_A^{(r \cdot (k \cdot (H(m)+r \cdot a)^{-1})~mod~l)} \]
%\[ g^k=g^{(h(m)) \cdot ((k^{-1} \cdot (H(m)+r \cdot a))^{-1} \mod l)} \cdot g^{a^{(r \cdot (k \cdot (H(m)+r \cdot a)^{-1})~mod~l)}} \]
%\[ g^k=g^{(h(m)) \cdot ((k^{-1} \cdot (H(m)+r \cdot a))^{-1} \mod l)} \cdot g^{a^{r}} \cdot g^{a^{(k \cdot (H(m)+r \cdot a)^{-1})~mod~l)}} \]

%\[ k=(h(m) \cdot k \cdot (H(m)+r \cdot a)^{-1} \mod l)+a^{(r \cdot w~mod~l)} \]
%\[ k=(h(m) \cdot k \cdot (H(m))^{-1}+(r \cdot a)^{-1} \mod l)+a^{u_2} \]
%\[ k=(k + (r \cdot a)^{-1} \mod l)+a^{(r \cdot w~mod~l)} \]

%\[ k=(k + r^{-1} \cdot a^{-1} \mod l)+a^{(r \cdot w~mod~l)} \]
%\[ k=(k + r^{-1} \cdot a^{-1} \mod l)+a^{(r \cdot (k^{-1} \cdot H(m)^{-1})+(r^{-1} \cdot a^{-1}))~mod~l)} \]

\subsection{}
We have:
\[
s = k^{-1}(H(m)+r \cdot a) \mod l
\]
With $k$, we get:
\[
ks=(H(m)+r \cdot a) \mod l
\]
Since we also know $H(m)$ (from $m$), we get:
\[
ks-H(m)=r \cdot a \mod l
\]
Then, we get $a$ by dividing with $r$:
\[
\frac{ks-H(m)}{r}=a \mod l
\]

\subsection{}
Using the previous result, we have to solve:
\[
\frac{k s_1 - H(m_1)}{r_1}
\&
\frac{k s_2 - H(m_2)}{r_2}
\pmod l
\]
\[
\frac{k 448 - 345}{365}
\&
\frac{k 969 - 567}{365}
\pmod l
\]
Uh-oh, looks like Alice used the same random number twice!
We solve for $k$.
\[
k 448 - 345
=
k 969 - 567
\pmod l
\]
\[
% TODO
k=679
\]


\end{document}
