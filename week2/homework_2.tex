\documentclass{article}
\usepackage{amsfonts}
\usepackage{amsmath}
\usepackage{hyperref}

\newenvironment{amatrix}[1]{%
  \left(\begin{array}{@{}*{#1}{c}|c@{}}
}{%
  \end{array}\right)
}

\author{Aram Verstegen \\ s4092368}
\title{Cryptography 1, Homework 2}

\begin{document}
\maketitle
\date

\section{}
There was some debate on how interpret the rotation operation; whether the rotation means simply a left shift (rotation without carry) or a rotation with carry.
The former is trivially forged, as only the last $\frac{32}{4} = 8$ bytes of the input are needed to compute the secret; everything before that would fall off on the left side of the 32-bit state.
We therefore continue on the assumption that rotation with carry is implied.
(For example, take the following 8-bits: 11110001. If we rotate that to the left with 4 bits, we would end up with 00011111.)

Majordomo actually uses XOR, rather than addition as described in the assignment.
Since both XOR and addition are associative and commutative, there is nothing stopping us from applying the algorithm in reverse to compute the 32-bit state for an unknown secret, based on a valid e-mail address / cookie pair.

The truncation to 32-bits is what makes the operation non-commutative, but because of the order in which the secret and address are concatenated, the information lost through truncation to 32 bits can be recovered from the address to get back to the 32-bit state which $h$ was in after processing the secret.

To subscribe God@heaven.af.mil to the mailing list the following steps need to be taken:
\begin{enumerate}
	\item The attacker subscribes himself to the mailing list, for example eve@evil.com;
	\item The attacker will then receive the legitimate 32-bits cookie as usual;
	\item The attacker applies $h()$ in reverse to recover the 32-bit state, let's call it $recovered\_state = h^{-1}(cookie, address)$ which, for each byte in the reversed e-mail address;
		\begin{itemize}
		\item Rotates the 32-bits number (cookie) to the right by 4 bits;
		\item Subtracts (or XORs) the byte value of the current character in the e-mail address;
		\item Keeps rotating and subtracting until it has subtracted and rotated the first character in the e-mail address;
		\item Now the attacker is left with a 32-bits value that is $h()$'s internal state for the secret string - in other words we have $h(secret)$.
		\end{itemize}
	\item While this does not give the attacker the secret, we have the 32-bit state from which to compute $h(secret||address)$ using $initial\_state = recovered\_state$ rather than starting from initial state zero.
	For this we will use the modified function $h_2$.
\[ h_2(address, recovered\_state) = h(secret||addresses) = h_2(secret||addresses,0) \]
	\item The attacker now proceeds to send an email subscribing the victim to the mailing list.
	\item After waiting for a bit (the average time for Majordomo to generate and send the cookie, which could be timed in step 1 and 2) the attacker replies with the previously calculated cookie in another forged email.
	\item Majordomo finds this cookie correct and activates the victim's subscription.
\end{enumerate}
	
There are several ways to combat this attack.
A simple fix would be to use the secret last, so $h(ak)$ instead of $h(ka)$.
The attacker cannot compute forwards beyond the e-mail address bytes because the secret is unknown to him and the result he obtains has been truncated to 32-bits so some crucial information is missing.
At least this would make the algorithm more strongly resemble a one-way hash function.

But this is still vulnerable to forgery, because the attacker may collect a number of e-mail address / cookie pairs and use a linear solver (or brute-force) to recover the secret by guessing the bits that fell off.
This search would have worst-case complexity of $2^n$ where $n$ is the length of the secret, assuming no more than 1 bit is truncated per consumed byte.
A much better way would be a proper keyed MAC algorithm like HMAC based on a cryptographically secure hash function that is still considered secure.

From the Majordomo developers mailing list we found out that the developers were actually quite concerned about the kind of attack this assignment is about.
Here is an entertaining comment from one of the threads about the gen\_cookie feature:
\begin{quotation}
	...
	Second, with the checksum algorithm used, how difficult is it going to be
	for an attacker to figure out the secret given a single or small handful of
	samples?  I did some quick experiments on my main machine (a 66 MHZ 486),
	and was able to check about 333 seeds/sec, which means it would take about
	5 months to do a brute-force search of a 32-bit seed space; however, would
	a brute force search actually be necessary?  Or is there a more elegant way
	of determining the seed?  I don't know, I'm not a checksum wiz, but I've
	seen approaches like this fail before, for reasons like this.  I wonder how
	long it would be before a "hack\_majordomo" script appeared in the
	underground to figure out a site's secret based on a few samples.  I don't
	\_know\_ that the algorithm has this problem, but I don't know that it
	doesn't, either, and I'm not in a position to make that determination.  The
	nice thing about cryptographic checksum algorithms like MD5 and SNEFRU is
	that a lot of people have spent a lot of energy demonstrating that they
	\_don't\_ have such weaknesses.
\end{quotation}
\cite{majordomo_thread}
(Note that SNEFRU was known to be broken as early as 1993. Also, somebody else in the thread proposed a method of hashing which is vulnerable to a hash-length extension attack.)

The script hack\_majordomo.py is attached.

\section{}


\bibliographystyle{plain}
\bibliography{\jobname} 

\end{document}
