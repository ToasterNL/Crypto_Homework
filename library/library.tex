\documentclass{article}
\usepackage{amsfonts}
\usepackage{amsmath}
\usepackage{hyperref}
\usepackage[left=2cm,right=2cm,top=2cm,bottom=2cm]{geometry}
%\newenvironment{amatrix}[1]{%
%  \left(\begin{array}{@{}*{#1}{c}|c@{}}
%}{%
%  \end{array}\right)
%}

\author{Mark Vijfvinkel \& Aram Verstegen \\ Radboud University}
\title{Cryptography 1 library}

\begin{document}
\maketitle
\date

\section{Fields}
Field: $\mathcal{F}_q$ \\

\begin{align*}
a \equiv b \Leftrightarrow n | (a-b) \\
\end{align*}
$a$ is congruent to b mod $n$.

\begin{align*}
	a+b \mod n &= (a \mod n) + (b \mod n) \mod n\\
\end{align*}

Ring: isomorphic to $\mathcal{Z}/n$ \\
Field: isomorphic to $\mathcal{Z}/p$ \\
(every nonzero element has an inverse).

\begin{align*}
	gcd(a,n) = 1 \Leftrightarrow \exists a^{-1} \mod n \\
\end{align*}
Coprimes to $n$ have an inverse mod $n$.

\begin{itemize}
	\item $\mathcal{F}_q$ denotes a field with $q$ elements.
	\item The smallest number $p>0$ with $pa = 0$ for $a$ in $\mathcal{F}_q$ is called the characteristic of $\mathcal{F}_q$.
	\item The characteristic of a finite field is a prime number.
	\item Fields with a prime number of elements are called prime fields.
	\item

	\item Cyclic groups of order $n$ are isomorphic to $\mathcal{Z}/n$.
	\item Cyclic groups are abelian (commutative).
\end{itemize}

\subsection{Ring with unity}

\subsection{Order of field}
The order (number of elements) of a finite field is $p^n$ where $p$ is the characteristic and $n$ is a positive integer limiting the number of coordinates. (Think of it as the number of bits for a field of characteristic 2.)

\subsection{Order of elements}

\subsection{Compute EGCD}
\subsection{Subfields of prime field}
\subsection{Factorize polynomial}
\subsection{Factorization pattern of polynomial}
\subsection{Compute number of irreducible polynomials}
\subsection{Show polynomial is irreducible (using Miller-Rabin test)}


\subsection{Show there is no polynomial of given degree}
\subsection{Smallest extension field in which polynomial factors completely into linear factors}
\subsection{Product of order of other polynomials of given degree}

\section{RSA}
\subsection{RSA encryption}
\subsubsection{Key generation}
\begin{align*}
primes~p, q \\
n &= pq \\
\phi(n) &= (p-1)(q-1) \\
e : gcd(\phi(n), e) &= 1~(\phi(n)~and~e~are~coprime) \\
d &\equiv e^{-1} \mod \phi(n) \\
\end{align*}
Compute $d$ using egcd.

\subsubsection{Encryption}
\begin{align*}
    c &= m^e \mod n \\
    m &= c^d \mod n \\
\end{align*}

\subsection{Signatures}
\begin{align*}
s &= H(m)^d \mod n \\
 s^e \mod n &\equiv H(m)~? \\
\end{align*}

\subsection{Faster RSA decryption knowing p and q}
\begin{align*}
v_1 &= c^d \mod p \\
v_2 &= c^d \mod q \\
Because~a^{p-1} \mod p = 1 \\
v_1 &= c^{d \mod p-1} \mod p \\
v_2 &= c^{d \mod q-1} \mod q \\
q_{inv} &= p^{-1} \mod q \\
u &= (v_2 - v_1)q_{inv} \mod q \\
if~u=0~add~q\\
c^d \mod n &= v_1 + up &= m \\
\end{align*}

\subsection{Batch RSA}

\section{ElGamal}
Uses multiplicative multiplicative group ${\mathcal{Z}/p}^{*}$.
\subsection{Key generation}
Choose random integer $m_p$, $1 \leq m_p \leq p-1$, compute $c_P=g^{m_P} \mod p$. \\
$c_P$ is public, $m_P$ secret.

\subsection{Encryption}
Choose random integer $r$. \\
$R=g^r$ \\
$S=m \cdot c_P^r$ \\
$c=(R,S)$ \\
$m=S/R^{m_P}$ \\

\subsection{Signatures}
% TODO make more foolproof
Choose random integer $r$ coprime to $p-1$, compute $R=g^r$. \\
Compute $S$ satisfying:
$m=m_P \cdot R + r \cdot S \mod p$ 
using egcd. \\
Verify using:
$g^m \equiv c_P^R R^S \mod p$ \\

\section{Integer factorization}
\subsection{Smooth numbers}
Smooth numbers factor completely into small prime numbers. A number is $B$-smooth if none of its factors are greater than $B$.

\subsection{Pollard's Rho}
We do a random walk on the group, using $x_0=1$, $x_{i+1} = x_i^2 + 1$.
We compute a list of these ${x_0, x_1, \ldots }$
We try to find cycles $x_2i - x_i$.
At some point $gcd(x_2i - x_i, n)$ is non-trivial and we have found a factor from $n$.

\begin{verbatim}
a0 = 1
b0 = 1
c = 1
a0 = (a0**2 + c) % n
b0 = (((b0**2 + c)**2) + c) % n
p = gcd(abs(b0 - a0), n)
if(p > 1):
        print ("Found a factor: " + str(p))
while (a0 != b0):
        a0 = (a0**2 + c) % n
        b0 = (((b0**2 + c)**2) + c) % n
        p = gcd(abs(b0 - a0), n)
        if(p > 1):
                print ("Found a factor: " + str(p))
                n = n / p
\end{verbatim}

\subsection{Pollard's p-1}
Given $n$ to factor, a smoothness parameter $S$, random $2 \leq a < n$: \\
$R = \Pi p \leq S, p^{\lfloor log_r(n) \rfloor}$ \\
$d = gcd(a^R-1, n)$ \\
$if~1 < d < n \rightarrow d~is~factor$ \\

\begin{verbatim}
def pminus1(n, S):
  a = randint(2,n)
  R = 1
  for q in primes(S):
    R = R*pow(q,floor(log(S,q)))
  d = gcd(pow(a,R,n)-1, n)
  if 1 < int(d) and int(d) < n:
    return d
  return False
\end{verbatim}

\subsection{Fermat test}
For $k$ rounds, values $1 \leq a < n-1$ \\
$gcd(a,n) = 1$ (coprimes) AND \\
$a^{n-1} \not\equiv 1 \pmod n \rightarrow n$ is composite,
otherwise $n$ is prime or Carmichael for $k$ rounds. \\

\section{Discrete logarithms}
\subsection{Baby-step giant-step}
$h = g^e \mod q$ \\
Compute group order $n$, $q-1$ for finite fields. \\
$m = \lfloor \sqrt{n} \rfloor$\\
$d = m^{-1} \pmod{q}$\\

Draw up a table for baby-steps, $g^i \mod q$.\\
\begin{tabular}{|c|c|c|c|c|c|c|c|c|}
\hline
$i = $ & 0 & 1 & 2 & 3 & 4 & 5 & 6 & \ldots \\
\hline
• & $1$ & $g$ & $g^2$ & $g^3$ & $g^4$ & $g^5$ & $g^6$ & \ldots \\
\hline
\end{tabular}

Draw up a table for giant-steps, $h \cdot d^j \mod q$ until you find a match for the baby-steps.\\
\begin{tabular}{|c|c|c|c|c|c|c|c|c|}
\hline
$j = $ & 0 & 1 & 2 & 3 & 4 & 5 & 6 & \ldots \\
\hline
• & $h$ & $h \cdot d$ & $h \cdot d^2$ & $h \cdot d^3$ & $h \cdot d^4$ & $h \cdot d^5$ & $h \cdot d^6$ & \ldots \\
\hline
\end{tabular} \\
$e = m \cdot j + i$

\subsection{Pohlig-Hellman}
When looking for $log_g(h) = k$ in the group $G$ generated by $g$ knowing $h = g^k mod q$.
Factor the group order $n$ into unique primes $p_i$.
Then $k$ is congruent to 


\subsection{Pollard's Rho}


\section{Elliptic curves}
\subsection{Edwards curves}
\subsubsection{Finding all affine points on curve}

\begin{enumerate}
\item Calculate table of squares. See table.
\item Calculate a table of inverses for $1,2,3,.., p-1$. This handy but not necessary.
\item Rewrite the Edwards Curve from $a \cdot x^2 + y^2 = 1 - d \cdot x^2 \cdot y^2$ to $y^2 = \frac{(1 - a \cdot x^2)}{(1-d \cdot x^2)}$. (If $a=1$, it is a `normal' Edwards curve and if $a\not= 1$ a twisted Edwards Curve.)
\item Enter all x-coordinates in $y^2 = \frac{(1 - a \cdot x^2)}{(1-d \cdot x^2)}$. From $x = 0$ until $x = p-1$. Take care by reducing $mod p$ all the time, this makes it easier when computing inverses (as you have that table already). When you have a fraction $\frac{m}{n}$ compute the inverse of $n$, using XGCD($n$,$p$). Then multiply the inverse with $m$. What you do is $m \cdot n^{-1} mod p$. Look for that result in the table at the row of $b^2$. If there is a square, you have found the coordinate (x,b) and (x,-b). If there is not a square then that point does not exist.

\end{enumerate}

%\begin{tabular}{|c|c|c|c|c|c|c|}
%\hline 
%$b$ & 0 & 1 & 2 & 3 & .. & $(p-1)/2$ \\ 
%\hline 
%$b^2$ & 0 & 1 & 4 & 9 & ..  & .. $mod p$ \\ 
%\hline 
%\end{tabular}

%% TODO

%Draft this table
\begin{tabular}{|c|c|c|c|c|c|c|c|c|}
\hline
$x $ & 0 & 1 & 2 & 3 & 4 & 5 & \ldots & (p-1)/2 \\
\hline
$x^2 $ & 0 & 1 & 4 mod p & 9 mod p & 16 mod p & 25 mod p & \ldots & $((p-1)/2)^2$ mod p \\
\hline
\end{tabular}
%b   | 0 | 1 | 2 | 3 .. | (p-1)/2
%b^2 | 0 | 1 | 4 mod p | 3^2 mod p

%then transform the equation $ax^2+y^2=1+dx^2y^2$
%to $y^2=(1-ax^2)/(1-dx^2)$

%Plug in all $x$ values and check whether the right hand side gives a square: if
%so you've found (x,y) and (x,-y) - if not there is no point with that
%x-coordinate.

%In case you encounter a division by 0 check the original equation. For
%computing the division by $(1-dx^2)$ you might want to have a list of inverses or
%a calculator that computes them for you and compute at the beginning of the
%exercise -- or bring a printed table with you. Also note that you can use more
%symmetries, in particular for a=1.

\subsubsection{Verify point is on curve}
\subsubsection{Compute points}
\subsubsection{Compute fractions}
\subsubsection{Compute discrete logarithm}
\subsubsection{Translate to Montgomery form}
\subsubsection{Translate to Weierstrass form}

\subsection{Weierstrass curves}
\subsubsection{Finding all points on curve}
\subsubsection{Verify point is on curve}
\subsubsection{Compute points}
\subsubsection{Compute fractions}
\subsubsection{Compute discrete logarithm}
\subsubsection{Translate to Montgomery form}
\subsubsection{Translate to Edwards form}

\end{document}
