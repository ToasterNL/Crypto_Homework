\documentclass{article}
\usepackage{amsfonts}
\usepackage{amsmath}
\usepackage{hyperref}
\usepackage[left=2cm,right=2cm,top=2cm,bottom=2cm]{geometry}
%\newenvironment{amatrix}[1]{%
%  \left(\begin{array}{@{}*{#1}{c}|c@{}}
%}{%
%  \end{array}\right)
%}

\author{Mark Vijfvinkel \& Aram Verstegen \\ Radboud University}
\title{Cryptography 1 library}

\begin{document}
\maketitle
\date

\section{Fields}
Field: $\mathcal{F}_q$ \\

\begin{align*}
a \equiv b \Leftrightarrow n | (a-b) \\
\end{align*}
$a$ is congruent to b mod $n$.

\begin{align*}
	a+b \mod n &= (a \mod n) + (b \mod n) \mod n\\
\end{align*}

Ring: isomorphic to $\mathcal{Z}/n$ \\
Field: isomorphic to $\mathcal{Z}/p$ \\
(every nonzero element has an inverse).

\begin{align*}
	gcd(a,n) = 1 \Leftrightarrow \exists a^{-1} \mod n \\
\end{align*}
Coprimes to $n$ have an inverse mod $n$.

\begin{itemize}
	\item $\mathcal{F}_q$ denotes a field with $q$ elements.
	\item The smallest number $p>0$ with $pa = 0$ for $a$ in $\mathcal{F}_q$ is called the characteristic of $\mathcal{F}_q$.
	\item The characteristic of a finite field is a prime number.
	\item Fields with a prime number of elements are called prime fields.
	\item

	\item Cyclic groups of order $n$ are isomorphic to $\mathcal{Z}/n$.
	\item Cyclic groups are abelian (commutative).
\end{itemize}

\subsection{Ring with unity}

\subsection{Order of field}
The order (number of elements) of a finite field is $p^n$ where $p$ is the characteristic and $n$ is a positive integer limiting the number of coordinates. (Think of it as the number of bits for a field of characteristic 2.)

\subsection{Order of elements}

\subsection{Compute EGCD}
\subsection{Subfields of prime field}
\subsection{Factorize polynomial}
\subsection{Factorization pattern of polynomial}
\subsection{Compute number of irreducible polynomials}
\subsection{Show polynomial is irreducible (using Miller-Rabin test)}


\subsection{Show there is no polynomial of given degree}
\subsection{Smallest extension field in which polynomial factors completely into linear factors}
\subsection{Product of order of other polynomials of given degree}

\section{RSA}
\subsection{RSA encryption}
\subsubsection{Key generation}
\begin{align*}
primes~p, q \\
n &= pq \\
\phi(n) &= (p-1)(q-1) \\
e : gcd(\phi(n), e) &= 1~(\phi(n)~and~e~are~coprime) \\
d &\equiv e^{-1} \mod \phi(n) \\
\end{align*}
Compute $d$ using egcd.

\subsubsection{Encryption}
\begin{align*}
    c &= m^e \mod n \\
    m &= c^d \mod n \\
\end{align*}

\subsection{Signatures}
\begin{align*}
s &= H(m)^d \mod n \\
 s^e \mod n &\equiv H(m)~? \\
\end{align*}

\subsection{Faster RSA decryption knowing p and q}
\begin{align*}
v_1 &= c^d \mod p \\
v_2 &= c^d \mod q \\
Because~a^{p-1} \mod p = 1 \\
v_1 &= c^{d \mod p-1} \mod p \\
v_2 &= c^{d \mod q-1} \mod q \\
q_{inv} &= p^{-1} \mod q \\
u &= (v_2 - v_1)q_{inv} \mod q \\
if~u=0~add~q\\
c^d \mod n &= v_1 + up &= m \\
\end{align*}

\subsection{Batch RSA}

\section{ElGamal}
Uses multiplicative multiplicative group ${\mathcal{Z}/p}^{*}$.
\subsection{Key generation}
Choose random integer $m_p$, $1 \leq m_p \leq p-1$, compute $c_P=g^{m_P} \mod p$. \\
$c_P$ is public, $m_P$ secret.

\subsection{Encryption}
Choose random integer $r$. \\
$R=g^r$ \\
$S=m \cdot c_P^r$ \\
$c=(R,S)$ \\
$m=S/R^{m_P}$ \\

\subsection{Signatures}
% TODO make more foolproof
Choose random integer $r$ coprime to $p-1$, compute $R=g^r$. \\
Compute $S$ satisfying:
$m=m_P \cdot R + r \cdot S \mod p$ 
using egcd. \\
Verify using:
$g^m \equiv c_P^R R^S \mod p$ \\

\section{Integer factorization}
\subsection{Smooth numbers}
Smooth numbers factor completely into small prime numbers. A number is $B$-smooth if none of its factors are greater than $B$.

\subsection{Pollard's Rho}
We do a random walk on the group, using $x_0=1$, $x_{i+1} = x_i^2 + 1$.
We compute a list of these ${x_0, x_1, \ldots }$
We try to find cycles $x_2i - x_i$.
At some point $gcd(x_2i - x_i, n)$ is non-trivial and we have found a factor from $n$.

\begin{verbatim}
a0 = 1
b0 = 1
c = 1
a0 = (a0**2 + c) % n
b0 = (((b0**2 + c)**2) + c) % n
p = gcd(abs(b0 - a0), n)
if(p > 1):
        print ("Found a factor: " + str(p))
while (a0 != b0):
        a0 = (a0**2 + c) % n
        b0 = (((b0**2 + c)**2) + c) % n
        p = gcd(abs(b0 - a0), n)
        if(p > 1):
                print ("Found a factor: " + str(p))
                n = n / p
\end{verbatim}

\subsection{Pollard's p-1}
Given $n$ to factor, a smoothness parameter $S$, random $2 \leq a < n$: \\
$R = \Pi p \leq S, p^{\lfloor log_r(n) \rfloor}$ \\
$d = gcd(a^R-1, n)$ \\
$if~1 < d < n \rightarrow d~is~factor$ \\

We choose a random exponent $R$ from the group, and see if $gcd(a^R -1, n)$
brings an interesting result.

\begin{verbatim}
def pminus1(n, S):
  a = randint(2,n)
  R = 1
  for q in primes(S):
    R = R*pow(q,floor(log(S,q)))
  d = gcd(pow(a,R,n)-1, n)
  if 1 < int(d) and int(d) < n:
    return d
  return False
\end{verbatim}

\subsection{Fermat test}
For $k$ rounds, values $1 \leq a < n-1$ \\
$gcd(a,n) = 1$ (coprimes) AND \\
$a^{n-1} \not\equiv 1 \pmod n \rightarrow n$ is composite,
otherwise $n$ is prime or Carmichael for $k$ rounds. \\

\section{Discrete logarithms}
\subsection{Baby-step giant-step}
$h = g^e \mod q$ \\
Compute group order $n$, $q-1$ for finite fields. \\
$m = \lfloor \sqrt{n} \rfloor$\\

Draw up a table for baby-steps, $g^i \mod q$.\\
\begin{tabular}{|c|c|c|c|c|c|c|c|c|}
\hline
$i $ & 0 & 1 & 2 & 3 & 4 & 5 & 6 & \ldots \\
\hline
$g^i \mod q$ & $1$ & $g$ & $g^2$ & $g^3$ & $g^4$ & $g^5$ & $g^6$ & \ldots \\
\hline
\end{tabular}

Draw up a table for giant-steps, $h \cdot g^(-jm) \mod q$ until you find a match for the baby-steps.\\
$d = {g^m}^{-1} \pmod{q}$\\
\begin{tabular}{|c|c|c|c|c|c|c|c|c|}
\hline
$j $ & 0 & 1 & 2 & 3 & 4 & 5 & 6 & \ldots \\
\hline
$(hg^{-m})^j \mod q$ & $h$ & $h \cdot d$ & $h \cdot d^2$ & $h \cdot d^3$ & $h \cdot d^4$ & $h \cdot d^5$ & $h \cdot d^6$ & \ldots \\
\hline
\end{tabular} \\
$e = m \cdot j + i$

\subsection{Pohlig-Hellman}
When looking for $log_g(h) = k$ in the group $G$ generated by $g$ knowing $h = g^k mod q$.
Factor the group order $n$ into unique primes $p_i$.
Then give congruence relations $k^{\frac{n}{f}} \equiv h^{\frac{n}{f}} \mod f$
which can be solved with CRT.

\subsection{Pollard's Rho}
Just like with Pollard's Rho method for factoring integers we are looking for
congruent points between the slow and the fast walk. These give rise to congruence
relations which we can solve with CRT or a linear solver.

\section{Elliptic curves}
\subsection{Edwards curves}
$ax^2+y^2 = 1 + dx^2y^2$

\subsubsection{Finding all affine points on curve}
\begin{enumerate}
\item Calculate table of squares. See table.
\item Calculate a table of inverses for $1,2,3,.., p-1$. This handy but not necessary.
\item Rewrite the Edwards Curve from $a \cdot x^2 + y^2 = 1 - d \cdot x^2 \cdot y^2$ to $y^2 = \frac{(1 - a \cdot x^2)}{(1-d \cdot x^2)}$. (If $a=1$, it is a `normal' Edwards curve and if $a\not= 1$ a twisted Edwards Curve.)
\item Enter all x-coordinates in $y^2 = \frac{(1 - a \cdot x^2)}{(1-d \cdot x^2)}$. From $x = 0$ until $x = p-1$. Take care by reducing $mod p$ all the time, this makes it easier when computing inverses (as you have that table already). When you have a fraction $\frac{m}{n}$ compute the inverse of $n$, using XGCD($n$,$p$). Then multiply the inverse with $m$. What you do is $m \cdot n^{-1} mod p$. Look for that result in the table at the row of $b^2$. If there is a square, you have found the coordinate (x,b) and (x,-b). If there is not a square then that point does not exist.

\end{enumerate}

%\begin{tabular}{|c|c|c|c|c|c|c|}
%\hline 
%$b$ & 0 & 1 & 2 & 3 & .. & $(p-1)/2$ \\ 
%\hline 
%$b^2$ & 0 & 1 & 4 & 9 & ..  & .. $mod p$ \\ 
%\hline 
%\end{tabular}

%% TODO

%Draft this table
\begin{tabular}{|c|c|c|c|c|c|c|c|c|}
\hline
$b $ & 0 & 1 & 2 & 3 & 4 & 5 & \ldots & (p-1)/2 \\
\hline
$b^2 $ & 0 & 1 & 4 mod p & 9 mod p & 16 mod p & 25 mod p & \ldots & $((p-1)/2)^2$ mod p \\
\hline
\end{tabular}
%b   | 0 | 1 | 2 | 3 .. | (p-1)/2
%b^2 | 0 | 1 | 4 mod p | 3^2 mod p

%then transform the equation $ax^2+y^2=1+dx^2y^2$
%to $y^2=(1-ax^2)/(1-dx^2)$

%Plug in all $x$ values and check whether the right hand side gives a square: if
%so you've found (x,y) and (x,-y) - if not there is no point with that
%x-coordinate.

%In case you encounter a division by 0 check the original equation. For
%computing the division by $(1-dx^2)$ you might want to have a list of inverses or
%a calculator that computes them for you and compute at the beginning of the
%exercise -- or bring a printed table with you. Also note that you can use more
%symmetries, in particular for a=1.

\subsubsection{Verify point is on curve}
Given a curve and a point, simply verify the equation holds by filling in x and y.
Compare equality modulo order of underlying field.

\subsubsection{Compute fractions}
At some point when during computation a fraction occurs, we have to use the inverse to convert the fraction into an integer.
E.g. Say we have the fraction $\frac{8}{50}$ in field $F_{71}$. Then we first compute the inverse of $50^{-1} mod 71 = 27$, using XGCD. Then $8*27 mod 71 = 3$. More generally, $\frac{a}{b}$ in field $F_q$ is $a*b^{-1} mod q$.

\subsubsection{Compute discrete logarithm}


\subsubsection{Translate Montgomery <--> Twisted Edwards Curve form}
A Montgomery Curve over field K with characteristic 2 in elliptic form : $M_{A,B}: Bv^2 = u^3 + Au^2 + u$.
A Twisted Edwards Curve in elliptic form: $E_{a,d}: ax^2 + y^2 = 1 + dx^2y^2$, with $a,d \in K \backslash \{0\}, a \not= d$

Theorem: Every twisted Edwards curve is birationally equivalent  to a Montgomery curve over $K$.
In  particular, the twisted Edwards curve $E_{a,d}$ is birationally equivalent to the Montgomery curve $M_{A,B}$ where $A = \frac{2(a+d)}{a-d}$, and $B = \frac{4}{a-d}$.


Mapping a point: $\psi\,:\,E_{a,d} \rightarrow M_{A,B}$.\\ 
$(x,y) \mapsto (u,v) = \left(\frac{1+y}{1-y},\frac{1+y}{(1-y)x}\right)$ \\

is a birational equivalence from $E_{a,d}$ to $M_{A,B}$, with inverse:\\
$\psi^{-1}$: $M_{A,B} \rightarrow E_{a,d}$ \\
$(u,v) \mapsto (x,y) = \left(\frac{u}{v},\frac{u-1}{u+1}\right),
a=\frac{A+2}{B}, d=\frac{A-2}{B}$



\subsubsection{Translate to Weierstrass form}

\subsection{Weierstrass curves}
\subsubsection{Finding all points on curve}
See Edwards Curve
\subsubsection{Verify point is on curve}
See Edwards Curve
\subsubsection{Compute fractions}
See Edwards Curve
\subsubsection{Compute discrete logarithm}
This really depends on the previous/related questions. The example that follows is from question 5c of the previous exam.
In question 5a and 5b we discover the property that a coordinate on the curve, might result in the scalar needed to double or add P with. E.g. 2P is (18,9), thus $\frac{18}{9}$ is $18*9^{-1} mod 71 = 2$, which is the same 2 as in 2P (note that the inverse is not necessary here as 18/9 is 2, but you might want this for a less neat fraction).
Thus given a point, in the case of 5c, (6,43). We could hypothesise the also $\frac{6}{43}$ applies here. Computing the integer of this, see Compute Fractions above, we obtain 15. Which would mean $15P = 8P + 4P + 3P$, these points we already computed in 5a and using addition rules for short Weierstrass curves, we indeed obtain the coordinate given. Thus the DL is indeed 15.
\subsubsection{Translate to Montgomery form}
\subsubsection{Translate to Edwards form}

\section{Examples}
From the exam of 28 January 2014
\subsection{RSA}
This problem is about RSA encryption

\subsubsection{a (1 point)}
Alice's public key is $(n, e) = (13589, 5)$. Encrypt the
message $m = 2801$ to Alice using schoolbook RSA (no padding).

$c=m^e \mod n$ \\
$c=2801^5 \mod 13589$ \\
\verb|c=pow(2801,5,13589)| \\
$c=8679$

\subsubsection{b (2 points)}
Let $p = 653$ and $q = 701$. Compute the public key
using $e = 3$ and the corresponding private key.

$n = pq = 653 \cdot 701$ \\
$n = 457753$ \\
$\phi(n) = (p-1)(q-1) = 456400$ \\
$d = e^{-1} \mod \phi(n)$ \\
$d = 3^{-1} \mod 456400$ \\
\verb|d = pow(3, -1, 456400)| \\
$d = 304267$ \\

$public = (n,e) = (457753, 3)$ \\
$private = (n,d)= (457753, 304267)$ \\


\subsubsection{c (3 points)}
Decrypt the message $c = 4839$ which was encrypted to your key under (b). Feel
free to use $p$ and $q$.

We know that while we \textit{can} use $p$ and $q$ here, we already have $d$ to let
us do it in one step.

$m=c^d \mod n$ \\
$m=4839^{304267} \mod 457753$ \\
\verb|m = pow(4839, 304267, 457753)| \\
$m = 1234$

\subsection{DLP}
This exercise is about computing discrete logarithms in some groups.
The order of 2 in $\mathcal{F}_{211}^{*}$is 210. Alice uses the group generated
by $g = 2$ for cryptography. Her public key is $g_c = 107$.
Your task is to compute $0 \leq k < 210$ with $2^k \equiv 107 \mod 211$ in the following two steps

\subsubsection{a (3 points)}
Compute k modulo 2 and modulo 3.

We are asked to find congruence relations between $g_c$ and the group from
which it came in order to solve them via the Chinese Remainder Theorem.

We are given that the order of 2 (the generator of $F_{211}^{*}$) is 210
meaning that it takes this many exponentiations to come full circle to
$2^{211} \mod 211 = 2$ (hence \textit{cyclic}).
The point now is that we can see how many subgroups are in our group by
factoring this order.
$210 = 2 \cdot 3 \cdot 5 \cdot 7$ \\

We should first check to see if $g_c^{\frac{210}{f}} \equiv 1 \mod n$
for the given two factors $f$, 2 and 3. And in this case they are.
For $f=2$ this means that $107^{105} \equiv 107^{210} \mod 211$.
In other words: the factor 2 does not affect the behavior of the subgroup which
$g_c$ 107 is in. From there we can conclude that $f=2$ was not involved in the
construction of $g_c$ and the same goes for $f=3$.
This allows us to go on to the next question, but what if $g_c^{\frac{210}{f}}
\not \equiv 1 \mod 211$?

Example where $g_c=181$

$n_1 = 210/2 = 105$ \\
$g^n_1 = 2^{105} \mod 211 = 210 = -1$ \\
${g_c}^n_1 = 181^{105} \mod 211 = 210 = -1$ \\
TODO finish
%So $g_c=g=2$ mod $

\subsubsection{b (6 points)}
Use the baby-step-giant-step algorithm to determine $k$.
Note, you can make use of the result obtained
under (a).

Given that we know $k \equiv 0 \mod 2$ and $k \equiv 0 \mod 3$, we can work in
subgroup $210/2/3=35$.
As long as we take care to exponentiate our generator and
$hg^{-1}$ appropriately this works completely transparently.
We find the number of steps $m$ to be taken with $m = round(sqrt(35)) = 6$
We draft the BSGS tables:

\begin{tabular}{|c|c|c|c|c|c|c|c|}
\hline
$j $ & 0 & 1 & 2 & 3 & 4 & 5 & 6  \\
\hline
$(2^6)^j \mod 211$ & $1$ & $64$ & $87$ & $82$ & $184$ & $171$ & $183$ \\
\hline
\end{tabular}

For the giant steps we can compute a new basis as $(2^6)^{-m} = (2^{36})^{-1} =
113$
so in stead of computing more inversions or multiplications we need only modular exponentiation.

\begin{tabular}{|c|c|c|c|c|c|c|c|}
\hline
$i $ & 0 & 1 & 2 & 3 & 4 & 5 & 6  \\
\hline
$107 \cdot (2^{36})^{-i} \mod 107$ & $107$ & $64$ & $58$ & $13$ & $203$ & $151$ & $183$ \\
\hline
\end{tabular}

This gives the result $k=m \cdot 1 + 1 = 6 \cdot 1 + 1 = 7$ meaning $k \equiv 7
\mod 35$.
This leads us to the one and only correct answer $k=42$, which we can check to
make sure.
$2^{42} \equiv 107 \mod 211$
\verb|pow(2,42,211)|

\subsection{Factoring}
This exercise is about factoring $n = 2014$. Obviously, 2 is a factor, so
the rest of the exercise is about factoring the remaining factor $m = 2014/2 =
1007$.

\subsubsection{a (5 points)}
Use Pollard’s rho method of factorization to find a factor of 1007.
Use starting point $x_0 = 1$, iteration function $x_{i+1} = x_i^2 + 1$ and
Floyd's cycle finding method, i.e. compute $gcd(x_{2i} - x_i , 1007)$ until
a non-trivial gcd is found.

We generate the random walk to get enough points.

$x_0 = 1$ \\
$x_1 = 2$ \\
$x_2 = 5$ \\
$x_3 = 26$ \\
$x_4 = 677$ \\
$x_5 = 145$ \\
$x_6 = 886$ \\
$x_7 = 544$ \\
$x_8 = 886$ \\

Try a few gcds:

$gcd(x2-x1, 1007) = gcd(3,   1007) = 1$ \\
$gcd(x4-x2, 1007) = gcd(672, 1007) = 1$ \\
$gcd(x6-x3, 1007) = gcd(860, 1007) = 1$ \\
$gcd(x8-x4, 1007) = gcd(209, 1007) = 19$

\small{
\begin{verbatim}
sage: def next(foo): return pow(foo,2,1007) +1 # floyd's cycle finding
sage: def da_range(this=0,total=10): return [next(this)] + da_range(next(this),total-1) if total > 0 else []
sage: xes=da_range(0,10)
sage: xes
[1, 2, 5, 26, 677, 145, 886, 544, 886, 544]
sage: [gcd(xes[2*x]-xes[x], 1007) for x in range(1,5)]
[1, 1, 1, 19]
\end{verbatim}
}

\subsubsection{b (2 points)}
Perform one round of the Fermat test with base
a = 2 to test whether 19 is prime.
What is the answer of the Fermat test?

We check $gcd(2,19) = 1$ (yes) \textbf{and} $2^{18} \mod 19 \not \equiv 1$ (no).
19 is not composite, so it's prime (or Carmichael).

\subsubsection{c (3 points)}
Use Pollard’s $p - 1$ factorization method to factor the number
$n = 1007$ with base $u = 2$ and exponent $2^3 \cdot 3^2$ .

$gcd(2^{8 \cdot 9} -1, 1007)$ \\
\begin{verbatim}
sage: gcd(2**(8*9) - 1, 1007)
19
\end{verbatim}



\end{document}
