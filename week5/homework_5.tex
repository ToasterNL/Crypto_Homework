\documentclass{article}
\usepackage{amsfonts}
\usepackage{amsmath}
\usepackage{hyperref}
\usepackage[left=2cm,right=2cm,top=2cm,bottom=2cm]{geometry}
%\newenvironment{amatrix}[1]{%
%  \left(\begin{array}{@{}*{#1}{c}|c@{}}
%}{%
%  \end{array}\right)
%}

\author{Mark Vijfvinkel \& Aram Verstegen \\ 0863002(s134674), s4092368 \\ Radboud University}
\title{Cryptography 1, Homework 5}

\begin{document}
\maketitle
\date

\section{Diffie-Hellman variant}
This DH variant uses addition in stead of multiplication (or multiplication in stead of exponentiation).
Even though it's in a different group, this is still a Diffie-Hellman problem.
We find the exponents $A$ and $B$ by trying candidate values $i$ until $g^i = h_A \pmod p$ or $g^i = h_B \pmod p$.
We can gain some speed by searching for both values while ranging over integer values for $i$ in one loop, and can do modular reduction on-the-fly rather than accumulating a large number.
This would also hold for the real DH case.

To calculate $Na$ we should do the following:

\[
Na \cdot g \equiv \mbox{234 mod 1009} 
\]
\[
(Na \cdot g) \mbox{ mod 1009} = \mbox{234 mod 1009}
\]
\[
(Na \cdot 123) \mbox{ mod 1009} = \mbox{234 mod 1009}
\]
To calculate $Nb$ we should do the following:
\[
Nb \cdot g \equiv \mbox{456 mod 1009} 
\]
\[
(Nb \cdot g) \mbox{ mod 1009} = \mbox{456 mod 1009}
\]
\[
(Nb \cdot 123) \mbox{ mod 1009} = \mbox{456 mod 1009}
\]

Using a Python script we obtain:
\[A=248\]
\[B=742\]


The shared secret by A is: $(456*248)\mbox{ mod 1009}$
The shared secret by B is: $(234*742)\mbox{ mod 1009}$

Which both result in $80$.
%\[Secret: 80\]

\section{Diffie-Hellman in $F_{2^4}$}
The order of the generators is $\varphi(2^4-1)$.
\[ \varphi(15) = (5-1)(3-1) = 8 \]

\[h_A = x^4 \]
\[h_B = x^7 \]
\[n = x^{11} \]
% TODO reduce to F_{2^4}

\section{Public Key Cryptosystem}
\subsection{3a}
\[M = (100 \cdot 103)-1=10299 \]
\[e = 39 \cdot 10299 + 100 = 401761\]
\[d = 61 \cdot 10299 + 103 = 525352\]
\[n = ((401761 \cdot 525352)-1)/10299 = 20493829\]

\[m'\equiv (525352*42) \mbox{ mod 20493829} = 1570955\]

To check our answer:

\[m\equiv (401761*1570955) \mbox{ mod 20493829} = 42\]

Which is indeed c.


\subsection{3b}
We can see that $n$ is integer by rewriting the given equations:

\[M=ab-1 \]
\[ed=(a'M+a)(b'M+b)\]
\[ed=a'Mb'M + a'Mb + ab'M + ab \]
\[ed-1=a'Mb'M + a'Mb + ab'M + ab-1 \]
\[n=ed-1/M=a'Mb' + a'b  + ab' + (M/M) \]
\[n=a'Mb' + a'b  + ab' + 1 \]
\[n=a'b'ab - a'b' + a'b  + ab' + 1 \]
Since $a, b, a'$ and $b'$ are all integers, $n$ is also integer.


\end{document}
