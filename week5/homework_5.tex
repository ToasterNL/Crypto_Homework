\documentclass{article}
\usepackage{amsfonts}
\usepackage{amsmath}
\usepackage{hyperref}
\usepackage[left=2cm,right=2cm,top=2cm,bottom=2cm]{geometry}
%\newenvironment{amatrix}[1]{%
%  \left(\begin{array}{@{}*{#1}{c}|c@{}}
%}{%
%  \end{array}\right)
%}

\author{Mark Vijfvinkel \& Aram Verstegen \\ 0863002(s134674), s4092368 \\ Radboud University}
\title{Cryptography 1, Homework 5}

\begin{document}
\maketitle
\date

\section{Diffie-Hellman variant}
This DH variant uses addition in stead of multiplication (or multiplication in stead of exponentiation).
Even though it's in a different group, this is still a Diffie-Hellman problem.
We find the exponents $A$ and $B$ by trying candidate values $i$ until $g^i = h_A \pmod p$ or $g^i = h_B \pmod p$.
We can gain some speed by searching for both values while ranging over integer values for $i$ in one loop, and can do modular reduction on-the-fly rather than accumulating a large number.
This would also hold for the real DH case.

Using a Python script we obtain:
\[A=248\]
\[B=742\]
\[Secret: 80\]

\section{Diffie-Hellman in $F_{2^4}$}
The order of the generators is $\varphi(2^4-1)$.
\[ \varphi(15) = (5-1)(3-1) = 8 \]

\[h_A = x^4 \]
\[h_B = x^7 \]
\[n = x^{11} \]
% TODO reduce to F_{2^4}

\section{Public Key Cryptosystem}
% TODO


\end{document}
