\documentclass{article}
\usepackage{amsfonts}
\usepackage{amsmath}
\usepackage{hyperref}
\usepackage[left=2cm,right=2cm,top=2cm,bottom=2cm]{geometry}
%\newenvironment{amatrix}[1]{%
%  \left(\begin{array}{@{}*{#1}{c}|c@{}}
%}{%
%  \end{array}\right)
%}

\author{Mark Vijfvinkel \& Aram Verstegen \\ 0863002(s134674), s4092368 \\ Radboud University}
\title{Cryptography 1, Homework 5}

\begin{document}
\maketitle
\date

\section{Diffie-Hellman variant}
This DH variant uses addition instead of multiplication (or multiplication instead of exponentiation).
Even though it's in a different group, this is still a Diffie-Hellman problem.
We find the exponents $n_A$ and $n_B$ by trying candidate values $i$ until $g \cdot i = h_A \pmod p$ or $g \cdot i = h_B \pmod p$.
We can gain some speed by searching for both values at the same time while ranging over integer values for $i$, and can do modular reduction on-the-fly rather than accumulating a large number.
This would also hold for the real DH case.

Using a Python script (attached) we obtain:
\[n_A=248\]
\[n_B=742\]

And we can verify this is correct:

%\[
%g \cdot A \equiv \mbox{234 mod 1009} 
%\]
%\[
%(Na \cdot g) \mbox{ mod 1009} = \mbox{234 mod 1009}
%\]
\[
(123 \cdot 248) \mbox{ mod 1009} = \mbox{234 mod 1009}
\]
%To calculate $Nb$ we should do the following:
%\[
%g \cdot B \equiv \mbox{456 mod 1009} 
%\]
%\[
%(Nb \cdot g) \mbox{ mod 1009} = \mbox{456 mod 1009}
%\]
\[
(123 \cdot 742) \mbox{ mod 1009} = \mbox{456 mod 1009}
\]


The secret for A is then: $(456 \cdot 248)\mbox{ mod 1009}$, 
and the secret for B is: $(234 \cdot 742)\mbox{ mod 1009}$.

Which both result in the shared secret $80$.
%\[Secret: 80\]

\section{Diffie-Hellman in $F_{2^4}$}
The order of the generators is $\varphi(2^4-1)$.
\[ \varphi(15) = (5-1)(3-1) = 8 \]

%This means that there are 8 generators which generate the entire group.
We simply fill in the values and reduce in $F(2)$.
\[h_A = x^4 \]
\[h_B = x^7 \]
\[n = x^{11} \]

We learned that the characteristic polynomial to reduce with cannot simply be derived for non-prime fields, but that we can start our tries with $x^n + ax + b$ for fields of size $p^n$.
Confirmation from sage:
\begin{verbatim}
sage: R=GF(2^4, name='x')
sage: R.modulus()
x^4 + x + 1
\end{verbatim}

Once we are sure we have the correct polynomial, we can do binary long division to reduce the value:

\begin{verbatim}
fedcba9876543210
    100000000000
    10011
    000110000000
       10011
       010110000
        10011
        00101000
          10011
          001110
\end{verbatim}

Leading to the answer $x^3 + x^2 + x$, which sage confirms.

\begin{verbatim}
sage: x=GF(2^4, name='x').gen()
sage: x^11
x^3 + x^2 + x
\end{verbatim}

\section{Public Key Cryptosystem}

\subsection{}
\[M = (100 \cdot 103)-1=10299 \]
\[e = 39 \cdot 10299 + 100 = 401761\]
\[d = 51 \cdot 10299 + 103 = 525352\]
\[n = ((401761 \cdot 525352)-1)/10299 = 20493829\]

First we check that it actually works:
\[m'\equiv (525352 \cdot 42) \mbox{ mod 20493829} = 1570955\]
\[m\equiv (401761 \cdot 1570955) \mbox{ mod 20493829} = 42\]

Then decrypt our given $c$:
\[ 401761 \cdot 42 \mbox{ mod 20493829} = 1570955 \]


\subsection{}
We can show that $n$ is integer by rewriting the given equations:

\[M=ab-1 \]
\[ed=(a'M+a)(b'M+b)\]
\[ed=a'Mb'M + a'Mb + ab'M + ab \]
\[ed-1=a'Mb'M + a'Mb + ab'M + ab-1 \]
\[n=ed-1/M=(a'Mb'M + a'Mb  + ab'M + M)/M \]
\[n=a'Mb' + a'b  + ab' + 1 \]
\[n=a'b'ab - a'b' + a'b  + ab' + 1 \]
Since $a, b, a'$ and $b'$ are all chosen as integers, $n$ is also integer.
%Since M represents the size of the group

The system works because $d$ and $e$ are eachother's multiplicative inverses modulo $n$.
The secret for the public key (118, 857) is the multiplicative inverse of 857 modulo 118, which is 99.
Unfortunately we weren't able to form a complete proof showing $ed = 1 \pmod n$ by hand.
An attempted proof follows:
\newpage

\[ e = a'M + a \]
\[ d = b'M + b \]
\[ ed = a'Mb'M + a'Mb + ab'M + ab \]
\[ n = ed-1/M \]
\[ ed-1 = a'Mb'M + a'Mb + ab'M + ab -1 \]
\[ ed-1 = a'Mb'M + a'Mb + ab'M + M \]
\[ ed = a'Mb'M + a'Mb + ab'M + M + 1 \]
\[ n =  a'b'M + a'b + ab' + 1 \]

Missing steps for modular reduction of $ed$ in $n$.

Applying sage:
\begin{verbatim}
sage: G=ZZ['a,b,ap,bp']
sage: a,b,ap,bp = G.gens()
sage: M=(a*b)-1
sage: e=ap*M+a
sage: d=bp*M+b
sage: n=G((e*d - 1) / M)
sage: e*d % n
1
\end{verbatim}

\[ ed = 1 \pmod n \]

\end{document}
