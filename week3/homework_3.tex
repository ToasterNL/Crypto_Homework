\documentclass{article}
\usepackage{amsfonts}
\usepackage{amsmath}
\usepackage{hyperref}

\newenvironment{amatrix}[1]{%
  \left(\begin{array}{@{}*{#1}{c}|c@{}}
}{%
  \end{array}\right)
}

\author{Mark Vijfvinkel \& Aram Verstegen \\ 0863002(s134674), s4092368 \\ Radboud University}
\title{Cryptography 1, Homework 3}

\begin{document}
\maketitle
\date

\section{Question 1: Lookup-tables}
AES starts with 128-bit block called the State-input, usually depicted as a 4x4-matrix. Each square of this matrix consist of 8-bits or 1 byte, represented in hexadecimal notation. 
The SubBytes-step substitutes these values with the values in the S-Box. This S-Box is in itself a lookup-table, it has a mapping for all possible hexadecimal values. 
When all of the original State-input is replaced with values from the S-Box we are left with a new 4x4-matrix. The ShiftRows-step will rotate rows two, three and four with one, two and three bytes respectively, but we will omit this for now.
The MixColumns-step ....


\section{Question 2: Modes of Operation}



%\bibliographystyle{plain}
%\bibliography{\jobname} 

\end{document}
