\documentclass{article}
\usepackage{amsfonts}
\usepackage{amsmath}
\usepackage{hyperref}
\usepackage[left=2cm,right=2cm,top=2cm,bottom=2cm]{geometry}
%\newenvironment{amatrix}[1]{%
%  \left(\begin{array}{@{}*{#1}{c}|c@{}}
%}{%
%  \end{array}\right)
%}

\author{Mark Vijfvinkel \& Aram Verstegen \\ 0863002(s134674), s4092368 \\ Radboud University}
\title{Cryptography 1, Homework 8}

\begin{document}
\maketitle
\date

\section{Pollard's Rho algorithm}
We have implemented Pollard's Rho algorithm as given.
We do a random walk until we hit a point where $t_i=r_i$ and where we can find inverses, then exploit this relation to find candidate solutions for a derived congruence relation.
\begin{verbatim}
sage: load pollards_rho.sage
Solving 3^a = 245 (mod 1013) using Pollard's Rho method
   t_i    a_i    b_i    r_i    a_j    b_j
     9      2      0     27      3      0
    27      3      0    243      5      0
    81      4      0    161      7      0
   243      5      0    666     28      0
   729      6      0    231     29      1
   161      7      0     53     31      1
   596     14      0    323     63      2
   666     28      0    589    126      5
   985     29      0    364    127      6
   231     29      1    108    128      7
   693     30      1    972    130      7
    53     31      1    947    262     14
   783     62      2    531    524     29
   323     63      2    280    525     30
  1003    126      4    161    526     31
   589    126      5    666     80    124
   459    126      6    231     81    125
   364    127      6     53     83    125
    36    127      7    323    167    250
   108    128      7    589    334    501
   324    129      7    364    335    502
   972    130      7    108    336    503
   890    131      7    972    338    503
   947    262     14    947    678   1006
Solving for a in:
 992a = 596 (mod 1012)
This has 4 solution(s)
Candidate 122 rejected
Found solution: a=375 in 0.004 s
\end{verbatim}

We're getting correct results fast using this implementation.

\section{Index calculus}
We have implemented the index calculus algorithm as given.
We pick random exponents of g in the group to find the relation between the exponents of the group's inputs and outputs.
If we find relationships that are invertible we store them in a matrix until we have a determined system.
With a solution for these relationships we have the $log_g$ for each of the factor base elements.
If we now find a factorization of any group element multiple of the public key $h$ we can use these factors to express h in terms of $g$ using the $log_g$ values we obtained.

\begin{verbatim}
sage: load index_calculus.sage
Solving 2^a = 281 (mod 1019) using index calculus method on base [2, 3, 5, 7, 11, 13]
Solved system:
[  1   0   0   0   0   0   1]
[  0   1   0   0   0   0 958]
[  0   0   1   0   0   0  10]
[  0   0   0   1   0   0 363]
[  0   0   0   0   1   0 756]
[  0   0   0   0   0   1 289]
Found solution a=311 in 0.060 s
\end{verbatim}

We have a working result, but much worse performance.
This is likely due to the required linear equation solver, but also our implementation is far from optimal in terms of memory use for the matrix system.
In any case, because we do a purely random search of the group it's less predictable how long it will take to form a fully determined linear system.

\end{document}
